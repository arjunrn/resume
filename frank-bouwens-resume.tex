% LaTeX source of my resume
% =========================

% Commented for easy reuse... ;)

% See the `README.md` file for more info.

% This file is licensed under the CC-NC-ND Creative Commons license.


% start a document with the here given default font size and paper size
\documentclass[10pt,a4paper]{article}

% include the `tex` instructions that takes care of loading packages and defining commands
% Copyright (c) 2012 Cies Breijs
%
% The MIT License
%
% Permission is hereby granted, free of charge, to any person obtaining a copy
% of this software and associated documentation files (the "Software"), to deal
% in the Software without restriction, including without limitation the rights
% to use, copy, modify, merge, publish, distribute, sublicense, and/or sell
% copies of the Software, and to permit persons to whom the Software is
% furnished to do so, subject to the following conditions:
%
% The above copyright notice and this permission notice shall be included in
% all copies or substantial portions of the Software.
%
% THE SOFTWARE IS PROVIDED "AS IS", WITHOUT WARRANTY OF ANY KIND, EXPRESS OR
% IMPLIED, INCLUDING BUT NOT LIMITED TO THE WARRANTIES OF MERCHANTABILITY,
% FITNESS FOR A PARTICULAR PURPOSE AND NONINFRINGEMENT. IN NO EVENT SHALL THE
% AUTHORS OR COPYRIGHT HOLDERS BE LIABLE FOR ANY CLAIM, DAMAGES OR OTHER
% LIABILITY, WHETHER IN AN ACTION OF CONTRACT, TORT OR OTHERWISE, ARISING FROM,
% OUT OF OR IN CONNECTION WITH THE SOFTWARE OR THE USE OR OTHER DEALINGS IN THE
% SOFTWARE.


% Some commands for making a LaTeX resume
% =======================================

% Commented ;)

% See the README.md file for more info



% \documentclass[10pt,a4paper]{article}  % i do this in the document itself


%%% LOAD AND SETUP PACKAGES

\usepackage[a4paper,margin=0.75in]{geometry}
\usepackage{mdwlist}   % to finetue lists with a inline heading and indented content (see Experiences)
\usepackage{multicol}  % for multiple column text
\usepackage{relsize}   % for \textscale, which I prefer over \sc (small caps), see my \acr command
\usepackage[english]{babel}
\hyphenation{Some-long-word}

\usepackage{hyperref}  % yups, URLs everwhere...
\usepackage{xcolor}  % ... and color them links
\definecolor{dark-blue}{rgb}{0.15,0.15,0.4}
\hypersetup{colorlinks,linkcolor={dark-blue},citecolor={dark-blue},urlcolor={dark-blue}}

\usepackage{ifxetex}
\ifxetex
  \usepackage{fontspec}
%  \setmainfont
%    [ ExternalLocation ,
%      Mapping          = tex-text ,
%      Numbers          = OldStyle ,
%      Ligatures        = {Common,Contextual} ,
%      BoldFont         = texgyrepagella-bold.otf ,
%      ItalicFont       = texgyrepagella-italic.otf ,
%      BoldItalicFont   = texgyrepagella-bolditalic.otf ]
%    {texgyrepagella-regular.otf}
  % Comment out the previous statement and uncomment the following line to use the
  % Linux Libertine font (it has nice lignatures).
  % Make sure to have the `ttf-linux-libertine` package installed on Ubuntu.
\setmainfont[Mapping=tex-text, Numbers=OldStyle, Ligatures={Common,Contextual}]{Linux Libertine Display}
  \usepackage[protrusion]{microtype}  % needs an experimental and impposible to find package for xetex
\else
  \usepackage{tgpagella}  % this case we lack lower case numbers, ligatures and some typographic niceties
  \usepackage[expansion,protrusion]{microtype}
\fi



%%% DOCUMENT WIDE STYLING

\pagestyle{empty}
\setlength{\tabcolsep}{0em}
\xspaceskip7pt  % some more spacing between sentences (use "i.e.\ " or "with SQL\@. " in case of errors)


%%% CUSTOM COMMANDS

% main title (name) with subtitle (date)
\newcommand*\maintitle[2]{\noindent{\LARGE \textbf{#1}}\ \ \ \emph{#2}}

% title for the root sections (experience, education, etc) of the resume
\newcommand*\roottitle[1]{\subsection*{#1}\vspace{-0.3em}\nopagebreak[4]}

% acr command, to quickly mark acronyms for special formatting
\newcommand*\acr[1]{\textscale{.85}{#1}}

% pretty bullet (created from a much smaller centerdot), \sbull contains its spacing
\newcommand*\bull{\raisebox{-0.365em}[-1em][-1em]{\textscale{4}{$\cdot$}}}
\newcommand*\sbull{\ \ \bull \ \ }

% it seems not to work when simply using \parindent...
\newlength{\newparindent}
\addtolength{\newparindent}{\parindent}

% a double \parindent...
\newlength{\doubleparindent}
\addtolength{\doubleparindent}{\parindent}
\addtolength{\doubleparindent}{\parindent}

% indentsection style, used for sections that aren't already in lists
% that need indentation to the level of all text in the document
\newenvironment{indentsection}%
{\begin{list}{}%
  {\setlength{\leftmargin}{\newparindent}\setlength{\parsep}{0pt}\setlength{\parskip}{0pt}\setlength{\itemsep}{0pt}\setlength{\topsep}{0pt}}%
}
{\end{list}}

% headerrow command, used for a new employer
\newcommand{\headedsection}[3]{\nopagebreak[4]\begin{indentsection}\item[]\textscale{1.1}{#1}\hfill#2#3\end{indentsection}\nopagebreak[4]}

% subheaderrow command, used for a new position
\newcommand{\headedsubsection}[3]{\nopagebreak[4]\begin{indentsection}\item[]\textbf{#1}\hfill\emph{#2}#3\end{indentsection}\nopagebreak[4]}

% body text (indented)
\newcommand{\bodytext}[1]{\nopagebreak[4]\begin{indentsection}\item[]#1\end{indentsection}\pagebreak[2]}

% \vspace variaties
\newcommand{\breakvspace}[1]{\pagebreak[2]\vspace{#1}\pagebreak[2]}
\newcommand{\nobreakvspace}[1]{\nopagebreak[4]\vspace{#1}\nopagebreak[4]}

% \spacedhrule a horizontal line with some vertical space before and after it
\newcommand{\spacedhrule}[2]{\breakvspace{#1}\hrule\nobreakvspace{#2}}

% \inlineheadsection command, used for a new employer
\newcommand{\inlineheadsection}[2]{\begin{basedescript}{\setlength{\leftmargin}{\doubleparindent}}\item[\hspace{\newparindent}\textbf{#1}]#2\end{basedescript}\vspace{-1.7em}}

% apo command, for an apostrophe that looks good on old style nums
\newcommand{\apo}{\raisebox{-.18ex}{'}{\hspace{-.1em}}}

% non space that allows line breaks
\newcommand*{\nsp}{\hskip0pt}

%%% MORE SPECIFIC COMMANDS

% CPP command (found it in some corner of the internet and decided to use it)
\newcommand{\CPP}{C\nolinebreak[4]\hspace{-.04em}\raisebox{.20ex}{\footnotesize\bf++} }

% KTurtle command :)
\newcommand{\KTurtle}{\acr{KT}urtle }



% % these are in the document itself:
%
% \begin{document}
% ...the document text...
% \end{document}




\begin{document}  % begin the content of the document
\sloppy  % this to relax whitespacing in favour of straight margins

\maintitle{Frank Bouwens}{May 30, 1991}  % title on top of the document

\nobreakvspace{0.3em}  % add some page break averse vertical spacing

% \noindent prevents paragraph's first lines from indenting
% \mbox is used to obfuscate the email address
% \sbull is a spaced bullet
% \href well..
% \\ breaks the line into a new paragraph
\noindent\href{mailto:frankkie12345@gmail.com}{frankkie12345\mbox{}@\mbox{}gmail.com}\sbull
\textsmaller{+}31.641594567\sbull
frankkie12345 \emph{(Skype)}\sbull
\href{http://www.linkedin.com/in/frankkie12345}{www.linkedin.com/in/frankkie12345}
\\
Jean Sibeliusstraat 111\sbull
3069\thinspace {\sc mj}\sbull
Rotterdam\sbull
The Netherlands

\spacedhrule{0.9em}{-0.4em}  % a horizontal line with some vertical spacing before and after

\roottitle{Summary}  % a root section title

\vspace{-1.3em}  % some vertical spacing
\begin{multicols}{2}  % open a multicolumn environment
\noindent \emph{Leergierige starter met passie voor mobiele applicaties, embedded software, open source software en Linux.}
\\
\\
Op de basisschool begon ik al met programmeren, namelijk met het programma Game Maker van \href{http://www.yoyogames.com}{yoyogames.com} (toen nog gamemaker.nl). De programmeertaal die ik daarin gebruikte, GML (Game Maker Language) was makkelijk om te leren. Het is namelijk een object-georienteerde taal met veel mogelijkheden die andere populaire programmeertalen ook ondersteunen. Een goede taal om mee te leren en te beginen met programmeren.
\\
\\
Op de middelbare school kwam ik in aanraking met PHP. Dat was heel anders dan GML. Maar wel anders op een goede manier. Ik heb vele uren met PHP en MySQL kunnen experimenteren. Het heeft niet veel echte websites opgeleverd, maar wel talloze proof-of-concept pagina's. Ik gebruik PHP nog regelmatig om websites dynamisch te maken en zelfs om REST-functies mee aan te sturen.
\\
\\
Op de hogeschool werd mij geleerd in Java te programmeren. Dit paste bij mij als een handschoen. Java werkt object georienteerd net als GML (en voor een deel ook PHP). Ik kon meteen aan de slag. Op dit moment kwam ik voor het eerst echt in aanraking met Threads. Iets waar ik nu dagelijks gebruik van maak.
\\
\\
Later op de hogeschool kwam er een project voorbij waarbij een Android applicatie moest worden ontwikkeld. Ik was direct verkocht.
Sindsdien werk ik dagelijks met Android, zowel als gebruiker als ontwikkelaar.
Ik heb inmiddels talloze applicaties ontwikkeld, waarvan enkele te vinden zijn in Google Play (de voormalige Android Market).
\end{multicols}

\spacedhrule{0em}{-0.4em}

\roottitle{Ervaring}

\headedsection  % sets the header for the section and includes any subsections
  {\href{http://www.veliq.com}{VeliQ}}
  {\textsc{Barendrecht, Nederland}} {%

  \headedsubsection  % sets the header for a subsection and contains usually body text
    {\acr{Android Developer}}
    {Feb \apo2012}
    {\bodytext{VeliQ maakt Mobile Device Management oplossingen voor de professionele markt. In MobiDM worden diverse andere system geintegreed om het systeem aantrekkelijk te maken voor grote bedrijven.
\\
Tijdens mij afstudeerstage bij VeliQ heb ik een API ontwikkeld waarmee andere bedrijven een mobiele beheerders portal kunnen maken. Deze was gebaseerd op een bestaande API binnen het pakket maar dan beter geschikt voor op mobiele apparaten. Zo is gebruik gemaakt van REST/JSON in plaat van SOAP. SOAP-calls zijn een stuk zwaarder, er zijn geen goede libraries voor en verbruiken meer data-verkeer. Ik heb proof-of-concept applicaties ontwikkeld om aan te tonen dat de API correct functioneerd. 
\\
Naast mijn afstudeerstage heb ik bij VeliQ ook gewerkt aan de Android client voor het systeem. Deze applicatie wordt gebruikt om de toestellen te beheren. De app dwingt op het toestel security-policies af. Zoals het verplicht stellen van een lockscreen-password. Ik heb onder andere gewerkt aan het inbouwen van een anti-virus module en een simpelere / gebruiksvriendelijkere  enrollment-procedure. 
}}

\headedsection
{\href{http://www.anewspring.nl}{aNewSpring}}
{\textsc{Rotterdam, Nederland}}
{
 \headedsubsection
 {\acr{Android Developer}}
 {Feb 2011}
 {\bodytext{aNewSpring maakt e-learning oplossingen voor opleiders. Het is een volledig systeem om cursussen te geven en studenten te begeleiden. Bij dit bedrijf heb ik een mobiele applicaties ontwikkeld die cursisten helpt met leren. De applicatie geeft de cursist iedere dag 10 vragen over de cursus. Vragen die incorrect zijn beantwoord komen de volgende dag terug. Net zolang to alle vragen correct zijn beantwoord. En zijn verschillende vraagtypes ondersteund, zoals multiple choice (\'{e}\'{e}n correct antwoord), multi choice (meerdere antwoorden), text (een woord is het antwoord) en hotspot (wijs plek aan op plaatje). De applicatie werkt samen met de backend-system van aNewSpring om dit mogelijk te maken. Tijdens het ontwikkelen van deze app heb ik samengewerkt met een andere Android programmeur en meerdere backen programmeurs.}}
}

\headedsection
{\href{http://www.themobilecompany.nl}{The Mobile Company}}
{\textsc{Amsterdam, Nederland}}
{
 \headedsubsection
 {\acr{Android Developer}}
 {Jan 2009-Jan 2011}
 {\bodytext{The Mobile Company ontwikkeld mobiele applicatie in opdracht van andere bedrijven. Ik heb meegeholpen bij Android projecten als: iTour360 en 9292ov Pro. Ook heb ik 9292ov voor BlackBerry daar ontwikkeld.
\\
iTour306 is een locatie gebaseerde tour-guide app. De app geeft namelijk een rondleiding door een stad, in de eerste versie was dat alleen Almere. De app vraagt de gebruiker om naar een bepaalde lokatie te lopen, om vervolgens over diverse bezienswaardigheden informatie voor te lezen. Toen deze app uitkwam waren smartphones nog niet zo heel gewoon. Niet iedereen had er een op zak. Om die reden werd app voorgeinstalleerd op HTC Wildfire toestellen. Om die toestellen vervolgens verhuren bij een VVV-kantoor. Om misbruik te voorkomen mocht de gebruiker niet buiten de applicatie komen. Dit is gedaan door een custom homescreen op de toestellen te zetten. Via dit homescreen kan alleen de iTour360 app weer worden geopend.
\\
Toen ik aan de 9292ov Pro app ging werken was deze net uitgebracht. De ontwikkelaar die de app gemaakt had was toen net weg bij het bedrijf, dit was namelijk een afstudeerder. Ik kreeg alleen de broncode. Er zaten bugs in het programma en die moesten eruit. Verder moesten er nieuwe functies worden ingebouwd. Het heeft een tijdje geduurd om code te begrijpen. Dit was omdat de ontwikkelaar geen Android app had ontwikkeld, maar een iOS app. En die broncode was door een converter gehaald om er een Android app van te maken. Hierdoor ontstonden onder andere methodes van honderden regels lang, waarbij de code na 16 indents pas begon. Na enkele maanden waren de belangrijkste bugs eruit, was een vertrektijden scherm ingebouwd en was er een widget toegevoegd.
\\
9292ov voor Blackberry werkte op dat moment alleen voor Blackberry versie 5. Mijn taak was om de applicatie te porten naar versie 6. Ik kreeg toen de broncode van de app, maar er klopte iets niet. Er was namelijk een library dit ontbrak. En het bedrijf kon mij die library niet leveren. Dit kwam omdat ex-medewerker die library had gemaakt maar nooit had ingeleverd. Ik had de keuze om die library te gaan ontwikkelen zonder enige kennis over wat die library moest doen of de applicatie opnieuw te ontwikkelen. Ik ben voor de laatste keuze gegaan. De applicatie was binnen enkele weken klaar. Het is gebaseerd op PhoneGap, om dit sneller te ontwikkelen was. De opdrachtgever had namelijk erg veel haast de applicatie te publiceren.
}}
}

}

\begin{center}
  \emph{Zie ook \href{http://www.linkedin.com/in/frankkie12345}{mijn profiel op Linkedin} voor een complete lijst van mijn werkervaring en aanbevelingen.}
\end{center}


\spacedhrule{-0.2em}{-0.4em}

\roottitle{Scholing}

\headedsection
  {Hogeschool Rotterdam}
  {\textsc{Rotterdam, Nederland}} {%
  \headedsubsection
    {Behaald Bachelor informatica}
    {2008 -- 2014}
    {\bodytext{Programmeren in Java, Android applicaties, etc.}}
}

\headedsection
  {Comenius College}
  {\textsc{Capelle aan den IJssel, Nederland}} {%
  \headedsubsection
    {\acr{HAVO} \textnormal{(Middelbare school, profiel: Natuur en Techniek)}}
    {2003 -- 2008} {}
}


\spacedhrule{0.5em}{-0.4em}

\roottitle{Vaardigheden}

\inlineheadsection  % special section that has an inline header with a 'hanging' paragraph
  {Technische specialisaties:}
  {Software design en implementatie (in een team). 
  Ik maak gebruik van talen als: Python/\nsp Java/\nsp en PHP. 
  Ik heb verstand van web-technologie\"en als:\ \acr{HTML+CSS}, \acr{XML}, \acr{REST}, \acr{SOAP} and JavaScript. Linux vaardigheden: bash, Apache, My\acr{SQL}, Postgres\acr{SQL}.}

\inlineheadsection
  {Talen:}
  {Nederlands \emph{(Moedertaal)}, Engels \emph{(Vloeiend)}, Duits \emph{(Beginner)}.}


\spacedhrule{1.6em}{-0.4em}

\roottitle{Intresses}

\inlineheadsection
  {In alfabetische volgorde:}
  {Android, cryptografie, Google-producten, muziek, open source,  software engineering, typograpfie (\LaTeX).}

\spacedhrule{1.6em}{-0.4em}

\roottitle{Projecten}

\inlineheadsection
  {ParcDroid:}
  {Schoolproject, mijn eerste Android applicatie. ParcDroid is een navigatie applicatie voor kinderen in \emph{De Efteling}. Er kan van tevoren een verzamelplaats worden gekozen. Als het kind de weg kwijt is geraakt helpt deze app om de weg naar deze lokatie te vinden. De app gebruikt daarvoor onder andere GPS en het kompas. Dit project heeft mij niet alleen meer over Android geleerd, maar ook over wiskunde en het Dijkstra-algoritme. (Het algoritme waarmee een route kan worden bepaald, wordt ook gebruikt door onder andere TomTom navigatiesystemen) ParcDroid heeft de eerste prijs in de wacht gesleept en was daarmee de beste applicatie van dat schooljaar.}
  
\inlineheadsection
{Slidepuzzle:}
{App ontwikkeld tijdens een programmeerwedstrijd. Het bedrijf \emph{Service2Media} heeft in 2012 een programeerwedstrijd georganiseerd. De bedoeling was het ontwikkelen van een native Android applicatie. De app maakt een foto, deze wordt opgedeeld in stukjes. De gebruiker moet volgens de stukjes doormiddel van schuiven (zoals de ouderwetse schuifpuzzel) het originele plaatje herstellen. Deze app is de 2e plaats bereikt.}

\inlineheadsection
{Tux Words (Lingo):}
{Hobby project. De app laat de gebruiker \emph{Lingo} spelen, zoals op TV. Deze applicatie is ontwikkeld om dat een vergelijkbare app niet op alle Android apparaten werkt. De concurrerende app werkt namelijk alleen op toestellen met een mdpi of hoger scherm. Kleine (vaak goedkopere) toestellen met een ldpi scherm zijn daarmee uitgesloten. Tux Words is gemaakt om op zoveel mogelijk apparaten te kunnen werken. Om die reden werkt het op alle geteste smartphones en tablets, maar ook op Google TV. Tux Words is mijn meest populaire app, met meer dan 20.000 downloads op de Google Play Store. Verder is deze app ook te vinden in de \href{http://apps.samsung.com/venus/topApps/topAppsDetail.as?productId=000000498917}{Samsung Apps Store} en de \href{http://www.amazon.com/FrankkieNL-Tux-Words-Lingo/dp/B0096M4AYU/}{Amazon Appstore}}

\inlineheadsection
{BAXY Launcher}
{Hobby project, custom homescreen voor OUYA game console. De \emph{OUYA} laat de gebruiker Android games spelen op een TV met behulp van een controller. Standaard toont OUYA de lijst van spellen in een niet te voorspellen volgorde. De volgorde wordt namelijk gebaseerd op volgorde van installatie, laatste update, laatst gebruikt en meest gebruikt. Het komt erop neer dat elke keer als een game wordt gestart, de volgorde van de lijst veranderd. Dit vinden de meeste gebruikers vervelend. Om die reden heb ik BAXY Launcher ontwikkeld. Deze zet de lijst van spellen namelijk in alfabetische volgorde. De launcher is verder uitgebreid met mogelijkheiden om te filteren. In tegenstelling tot de standaard launcher kan BAXY ook widgets weergeven en gebruikmaken van een livewallpaper. Zie \href{http://ouyaforum.com/showthread.php?4436-BAXY-Custom-Launcher}{ouyaforum.com/showthread.php?4436} voor meer informatie en screenshots.}

\inlineheadsection
{StylusKeyboard}
{Hobby project, geinspireerd door de \emph{Samsung Galaxy Note} en de app \emph{Palm Graffiti}. Deze toetsenbord implementatie laat de gebruiker, met een stylus, letters schrijven (tekenen) in plaats van te typen. De app maakt gebruik van zogenaamde 'gestures'. De gebruiker kan eenvoudig nieuwe tekens met bijbehorende tekening toevoegen. Deze applicatie is niet alleen te vinden in de \href{https://play.google.com/store/apps/details?id=nl.frankkie.styluskeyboard}{Google Play Store}, maar ook op de \href{http://apps.samsung.com/venus/topApps/topAppsDetail.as?productId=000000499470}{Samsung Apps Store}.}


\inlineheadsection{Google Glass}{
Via een collega heb ik een week lang met \emph{Google Glass} mogen experimenteren. In deze week heb ik een applicatie ontwikkeld die gebruik maakt van de camera en gezichten detecteerd. De applicatie herkent niet \emph{wie} hij ziet, maar alleen dat er een gezicht is.
}

\inlineheadsection
{Live Notifications voor Sony LiveView Smartwatch}
{
Smartwatches zijn voornamelijk bedoeld om notificaties van de telefoon weer te geven. Tot mijn verbazing werkte deze functie niet correct op de \emph{Sony LiveView} smartwatch. Alleen apps die zijn aangesloten bij Sony (en dus Sony SDK gebruiken) konden notificaties naar het horloge zenden. notificaties van Apps zoals \emph{WhatsApp} kwamen niet aan op het horloge. Om die reden heb ik een app gemaakt die \emph{alle} notificaties op het toestel doorstuurt naar het horloge. Op oudere versie van Android was het niet zomaar toegestaan om notificaties af te vangen. Om die reden gebruikt de app op Android 4.2 en lager de Accessibility-service. Zie ook:  \href{https://play.google.com/store/apps/details?id=nl.frankkie.livenotifications}{Google Play Store}
}

\inlineheadsection{Apps voor Omate Truesmart Smartwatch}{
Via een \href{https://www.kickstarter.com/projects/omate/omate-truesmart-water-resistant-standalone-smartwa}{Kickstarter} project ben ik aan een \href{http://www.omate.com/}{Omate} Smartwatch gekomen. Ik kwam er al snel achter dat het horloge wat gebreken had. Met behulp van apps heb ik de voor mij belangrijkste problemen opgelost. Het horloge telt niet door als hij uit staat. Iets wat redelijk vaak gebeurt omdat de accu vrij snel leeg gaat. Gaat het horloge om 12:00 uit, en wordt hij om 13:00 weer aangezet, dan meld het horloge dat het 12:00 is. Het horloge heeft wel ingebouwde mogelijkheden om de tijd te synchroniseren, maar deze vereisen een simkaart of een sterk gps signaal. Om die reden heb ik een applicatie ontwikkeld die de tijd synchroniseerd via een NTP-server op het internet. Zie ook \href{https://play.google.com/store/apps/details?id=nl.frankkie.ontp}{Google Play Store}. Een ander probleem van het horloge is het ontbreken van volume-knoppen. Het horloge bevat wel een mp3-speler applicatie, maar die staat altijd even hard. Een simpele applicatie met volume-knoppen lost dit probleem op.
}

\begin{center}
  \emph{Zie ook \href{http://frankkienl.github.com/}{Github} en \href{https://play.google.com/store/apps/developer?id=FrankkieNL}{de Google Play Store} voor een uitgebreidere lijst van mijn projecten.}
\end{center}

\end{document}
