% LaTeX source of my resume
% =========================

% Commented for easy reuse... ;)

% See the `README.md` file for more info.

% This file is licensed under the CC-NC-ND Creative Commons license.


% start a document with the here given default font size and paper size
\documentclass[10pt,a4paper]{article}

% include the `tex` instructions that takes care of loading packages and defining commands
% Copyright (c) 2012 Cies Breijs
%
% The MIT License
%
% Permission is hereby granted, free of charge, to any person obtaining a copy
% of this software and associated documentation files (the "Software"), to deal
% in the Software without restriction, including without limitation the rights
% to use, copy, modify, merge, publish, distribute, sublicense, and/or sell
% copies of the Software, and to permit persons to whom the Software is
% furnished to do so, subject to the following conditions:
%
% The above copyright notice and this permission notice shall be included in
% all copies or substantial portions of the Software.
%
% THE SOFTWARE IS PROVIDED "AS IS", WITHOUT WARRANTY OF ANY KIND, EXPRESS OR
% IMPLIED, INCLUDING BUT NOT LIMITED TO THE WARRANTIES OF MERCHANTABILITY,
% FITNESS FOR A PARTICULAR PURPOSE AND NONINFRINGEMENT. IN NO EVENT SHALL THE
% AUTHORS OR COPYRIGHT HOLDERS BE LIABLE FOR ANY CLAIM, DAMAGES OR OTHER
% LIABILITY, WHETHER IN AN ACTION OF CONTRACT, TORT OR OTHERWISE, ARISING FROM,
% OUT OF OR IN CONNECTION WITH THE SOFTWARE OR THE USE OR OTHER DEALINGS IN THE
% SOFTWARE.


% Some commands for making a LaTeX resume
% =======================================

% Commented ;)

% See the README.md file for more info



% \documentclass[10pt,a4paper]{article}  % i do this in the document itself


%%% LOAD AND SETUP PACKAGES

\usepackage[a4paper,margin=0.75in]{geometry}
\usepackage{mdwlist}   % to finetue lists with a inline heading and indented content (see Experiences)
\usepackage{multicol}  % for multiple column text
\usepackage{relsize}   % for \textscale, which I prefer over \sc (small caps), see my \acr command
\usepackage[english]{babel}
\hyphenation{Some-long-word}

\usepackage{hyperref}  % yups, URLs everwhere...
\usepackage{xcolor}  % ... and color them links
\definecolor{dark-blue}{rgb}{0.15,0.15,0.4}
\hypersetup{colorlinks,linkcolor={dark-blue},citecolor={dark-blue},urlcolor={dark-blue}}

\usepackage{ifxetex}
\ifxetex
  \usepackage{fontspec}
%  \setmainfont
%    [ ExternalLocation ,
%      Mapping          = tex-text ,
%      Numbers          = OldStyle ,
%      Ligatures        = {Common,Contextual} ,
%      BoldFont         = texgyrepagella-bold.otf ,
%      ItalicFont       = texgyrepagella-italic.otf ,
%      BoldItalicFont   = texgyrepagella-bolditalic.otf ]
%    {texgyrepagella-regular.otf}
  % Comment out the previous statement and uncomment the following line to use the
  % Linux Libertine font (it has nice lignatures).
  % Make sure to have the `ttf-linux-libertine` package installed on Ubuntu.
\setmainfont[Mapping=tex-text, Numbers=OldStyle, Ligatures={Common,Contextual}]{Linux Libertine Display}
  \usepackage[protrusion]{microtype}  % needs an experimental and impposible to find package for xetex
\else
  \usepackage{tgpagella}  % this case we lack lower case numbers, ligatures and some typographic niceties
  \usepackage[expansion,protrusion]{microtype}
\fi



%%% DOCUMENT WIDE STYLING

\pagestyle{empty}
\setlength{\tabcolsep}{0em}
\xspaceskip7pt  % some more spacing between sentences (use "i.e.\ " or "with SQL\@. " in case of errors)


%%% CUSTOM COMMANDS

% main title (name) with subtitle (date)
\newcommand*\maintitle[2]{\noindent{\LARGE \textbf{#1}}\ \ \ \emph{#2}}

% title for the root sections (experience, education, etc) of the resume
\newcommand*\roottitle[1]{\subsection*{#1}\vspace{-0.3em}\nopagebreak[4]}

% acr command, to quickly mark acronyms for special formatting
\newcommand*\acr[1]{\textscale{.85}{#1}}

% pretty bullet (created from a much smaller centerdot), \sbull contains its spacing
\newcommand*\bull{\raisebox{-0.365em}[-1em][-1em]{\textscale{4}{$\cdot$}}}
\newcommand*\sbull{\ \ \bull \ \ }

% it seems not to work when simply using \parindent...
\newlength{\newparindent}
\addtolength{\newparindent}{\parindent}

% a double \parindent...
\newlength{\doubleparindent}
\addtolength{\doubleparindent}{\parindent}
\addtolength{\doubleparindent}{\parindent}

% indentsection style, used for sections that aren't already in lists
% that need indentation to the level of all text in the document
\newenvironment{indentsection}%
{\begin{list}{}%
  {\setlength{\leftmargin}{\newparindent}\setlength{\parsep}{0pt}\setlength{\parskip}{0pt}\setlength{\itemsep}{0pt}\setlength{\topsep}{0pt}}%
}
{\end{list}}

% headerrow command, used for a new employer
\newcommand{\headedsection}[3]{\nopagebreak[4]\begin{indentsection}\item[]\textscale{1.1}{#1}\hfill#2#3\end{indentsection}\nopagebreak[4]}

% subheaderrow command, used for a new position
\newcommand{\headedsubsection}[3]{\nopagebreak[4]\begin{indentsection}\item[]\textbf{#1}\hfill\emph{#2}#3\end{indentsection}\nopagebreak[4]}

% body text (indented)
\newcommand{\bodytext}[1]{\nopagebreak[4]\begin{indentsection}\item[]#1\end{indentsection}\pagebreak[2]}

% \vspace variaties
\newcommand{\breakvspace}[1]{\pagebreak[2]\vspace{#1}\pagebreak[2]}
\newcommand{\nobreakvspace}[1]{\nopagebreak[4]\vspace{#1}\nopagebreak[4]}

% \spacedhrule a horizontal line with some vertical space before and after it
\newcommand{\spacedhrule}[2]{\breakvspace{#1}\hrule\nobreakvspace{#2}}

% \inlineheadsection command, used for a new employer
\newcommand{\inlineheadsection}[2]{\begin{basedescript}{\setlength{\leftmargin}{\doubleparindent}}\item[\hspace{\newparindent}\textbf{#1}]#2\end{basedescript}\vspace{-1.7em}}

% apo command, for an apostrophe that looks good on old style nums
\newcommand{\apo}{\raisebox{-.18ex}{'}{\hspace{-.1em}}}

% non space that allows line breaks
\newcommand*{\nsp}{\hskip0pt}

%%% MORE SPECIFIC COMMANDS

% CPP command (found it in some corner of the internet and decided to use it)
\newcommand{\CPP}{C\nolinebreak[4]\hspace{-.04em}\raisebox{.20ex}{\footnotesize\bf++} }

% KTurtle command :)
\newcommand{\KTurtle}{\acr{KT}urtle }



% % these are in the document itself:
%
% \begin{document}
% ...the document text...
% \end{document}




\begin{document}  % begin the content of the document
\sloppy  % this to relax whitespacing in favour of straight margins

\maintitle{Arjun Naik}{October 29, 1987}  % title on top of the document

\nobreakvspace{0.3em}  % add some page break averse vertical spacing

% \noindent prevents paragraph's first lines from indenting
% \mbox is used to obfuscate the email address
% \sbull is a spaced bullet
% \href well..
% \\ breaks the line into a new paragraph
\noindent\href{mailto:Arjun.RN@gmail.com}{Arjun.RN\mbox{}@\mbox{}gmail.com}\sbull
\textsmaller{+}49.17657886406
\sbull arjun-naik \emph{(Skype)}
\\
\sbull
\href{https://de.linkedin.com/in/arnaik}{https://de.linkedin.com/in/arnaik}
\\
Gret-Palucca-Stra{\ss}e 9, Whng. 1214\sbull
01069\sbull
Dresden\sbull
Germany

\spacedhrule{0.9em}{-0.4em}  % a horizontal line with some vertical spacing before and after

\roottitle{Summary}  % a root section title

\vspace{-1.3em}  % some vertical spacing
\begin{multicols}{2}  % open a multicolumn environment
\noindent \emph{A organised, meticulous, and determined software engineering with a focus on building distributed systems.}
\\
\\
It has been 10 years since I picked up a book on C programming. Since then I have been led deeper into the world of software development. My discovery of Python was the second biggest turning point. This sparked my interest in Web applications and subsequently Distributed Systems. 
\\
After obtaining a Bachelor's degree in Computer Science and Engineering I spent 2 years working at a consultancy. Here I built the front-end for India's largest pizza chain. After that I moved to a startup where I built web applications and got interested in Android development. After working for 2 years in the industry I decided to continue my education in Germany. Studying at TU Dresden in the Systems Engineering chair has given me a unique opportunity to participate is several diverse research projects.

I am also deeply passionate about open-source and I try to contribute whenever I have the opportunity. I always encourage people to use and develop for open-source.

\\

\end{multicols}

\spacedhrule{0em}{-0.4em}

\roottitle{Experience}

\headedsection  % sets the header for the section and includes any subsections
{\href{http://www.netbramha.com}{Netbramha Studios}}
{\textsc{Bangalore, India}}
{
  \headedsubsection  % sets the header for a subsection and contains usually body text
    {\acr{Web Developer}}
    {Oct 2010 - May 2011}
    {\bodytext{Netbramha Studios is a consultancy which develops beautiful and content intensive websites for their clients. I was hired right out of college to handle the technical aspects of maintaining these websites as well as minor front-end development. Some projects were built using third-party as well as in-house frameworks. My work mostly dealt with maintainence and improvement of these applications. For the front-end development I used jQuery and Dojo as well as some basic layout and styling with \acr{HTML+CSS}. As my responsibilities grew I was handed the task of developing a complete web-based front end for one of India's largest pizza delivery chains.}}
}

\headedsection
{\href{http://www.notiphi.com}{Locus Labs}}
{\textsc{Bangalore, India}}
{
 \headedsubsection
 {\acr{Web and Mobile Developer}}
 {Jun 2011 - August 2012}
 {\bodytext{
 Locus Labs was at that time a seeded startup which developed apps for the Singapore market. The target audience were saavy shopper. The application {\href{http://www.notikum.com}{Notikum}} found offers for shoppers based on their profile. This involved collecting data from hundreds of websites. I was involved in the team which scraped the data using Scrapy(Python). This data was later indexed using Solr. I did this while concurrently developing and maintaining the website for this application. The surge in mobile apps necessitated that an Android app be development. I took over the development of this app after some initial work. 

After the startup pivoted I was given further responsibilities. I was put in charge of the development of the new product {\href{http://www.notiphi.com}{Notiphi}}. Location based advertising was the primary goal of this service. Advertisement were delivered to customers whenever they were in the vicinity of a pre-selected area. I designed and implemented the initial prototype with Tornado and PostgreSQL and also the relevant cloud infrastructure. I also leveraged my previous Android experience to develop the Android SDK which could be used by third-party developers to integrate this service into their apps.}}
}

\headedsection
{\href{http://dmfs.org/}{DMFS}}
{\textsc{Dresden, Germany}}
{
 \headedsubsection
 {\acr{Application Developer}}
 {Jan 2013 - March 2014}
 {\bodytext{
    DMFS develops data sync applications for the Android platform. I was responsible for the development of the {\href{https://play.google.com/store/apps/details?id=org.dmfs.tasks}{Tasks}} app which uses the CalDav protocol to sync calendar data. This app was subsequently open-sourced and available on {\href{https://github.com/dmfs/tasks}{Github}}.I was also part of the team which developed an Android based {\href{https://play.google.com/store/apps/details?id=com.schedjoules.calstore}{Calendar Store}}.
    
    But my primary role was as a Django developer. I developed a web application for the bulk-retail of mobile applications. The application used OpenSSL cryptography to sign and verify the source of the applications. Only applications with a valid key issued from the website could be activated. The Android SDK which integrates with the third-party app to provide this functionality was also developed by me.
}}
}

\headedsection
{\href{http://www.inquence.com/}{INQUENCE GmbH}}
{\textsc{Dresden, Germany}}
{
 \headedsubsection
 {\acr{Systems Developer}}
 {Mar 2014 - Jul 2014}
 {\bodytext{
INQUENCE develops a complete hardware and software solution for office for the the storage and documentation of the all types of documentation. A proprietary algorithm is used to quick search and retrieve the documents on demand. I was responsible for the maintenance of the web-frontend for the application and well as the addition of new features. The software solution consisted of several service which were responsible for converting, indexing and storing the data. I developed a monitoring mechanism using Monit which kept track for all the individual hardware units and monitored for performance changes after updates. 
}}
}

\headedsection
{\href{http://tu-dresden.de/die_tu_dresden/fakultaeten/fakultaet_informatik/sysa/se?set_language=en&cl=en}{TU Dresden}}
{\textsc{Dresden, Germany}}
{
 \headedsubsection
 {\acr{Wissenschaftliche Hilfskraft}}
 {Mar 2013 - Present}
 {\bodytext{
I work on the {\href{http://tu-dresden.de/die_tu_dresden/fakultaeten/fakultaet_informatik/sysa/se/srex}{SREX(Secure Remote Execution)}} project. I worked with a team which developed a translator for C programs. The C program were translated into a subset of C which can be encoded for redundant operations to ensure correct execution even in the event of bit-flips or other random transient errors.

More recently I have been working with distributed coordination services like ZooKeeper and Consul. I have been evaluating the impact of the respective protocols(ZAB and Raft). One of my duties also includes the implementation of new features to Kazoo, a Python client library for Zookeeper.
}}
}

\begin{center}
  \emph{Please consult my \href{https://de.linkedin.com/in/arnaik}{LinkedIn Profile} for references.}
\end{center}


\spacedhrule{-0.2em}{-0.4em}

\roottitle{Education}

\headedsection
  {BM Sreenivasaiah College of Engineering, Vivesvararya Technological University}
  {\textsc{Bangalore, India}} {%
  \headedsubsection
    {Bachelor of Computer Science and Engineering}
    {2006 -- 2010}
    {\bodytext{
        After securing a rank of 446 among more than 200,000 other University examination students I was granted a scholarship by the state government. I was also one of editors of the College newsletter and also a regular contributor. My major was in Computer Science with a focus on Computer Architecture. I also took elective courses in Compiler Design and Discrete Mathematics. For my final year project I developed a Wi-Fi enable robot which could be controlled over the internet. It also had a plethora of sensors which guided in navigation and control.
    }}
}

\headedsection
  {Techniche Universit{\"a}t Dresden}
  {\textsc{Dresden, Germany}} {%
  \headedsubsection
    {Master of Science, Distributed Systems Engineering}
    {2012 -- 2014}
    {\bodytext{
        I completed courses in Security and Cryptography, Software Fault Tolerance, Dependable Systems and various Labs. Under the guidance of Prof. Dr. Christof Fetzer I am in the process of completing my thesis. I have implemented and evaluated a partitioned Zookeeper service. This is used to increase the throughput of Zookeeper without the negative effects of horizontal scaling.
    }}
}


\spacedhrule{0.5em}{-0.4em}

\roottitle{Skills}

\inlineheadsection  % special section that has an inline header with a 'hanging' paragraph
  {Technical Specialities:}
  {My biggest technical skill is that I have an interest in a variety of programming language. I quickly learn and put to use any programming language which is required for the task at hand. However, Python is my favourite and go-to language for all occasions. I have used many Python based libraries and frameworks over the course of career. Django, Scrapy, Google AppEngine, Tornado, Fabric, pyOpenSSL are ones which I know best. I have also a good understanding of Frontend web technologies: \acr{HTML+CSS}, JavaScript(jQuery, Angular.js). I have also some experience with developing web applications using PHP, Ruby and node.js. All my projects are managed through Git or Mercurial. I also have substantial experience developing beautiful native Android applications. Most of my knowledge of Java comes from developing Android applications. Recently I have dabbled in Go and Haskell to solve some concurrency based problems for my courses. I also have some experience using and tuning databases like MySQL, PostgreSQL, MongoDB and Redis. I have also been using Linux both professionally and for personal use. I have a good understanding of the architecture of Linux as well as how to use it effectively.
}

\inlineheadsection
  {Natural Languages:}
  {English \emph{(Bilingual Proficiency)}, Hindi \emph{(Native)}, Kannada \emph{(Native)},German \emph{(Limited proficiency)}.}


\spacedhrule{1.6em}{-0.4em}

\roottitle{Interests}

\inlineheadsection
  {Non-exhaustive and in no particular order:}
  {Science-Fiction, photography, hardware hacking, travelling, good coffee}

\end{document}
