% LaTeX source of my resume
% =========================

% Commented for easy reuse... ;)

% See the `README.md` file for more info.

% This file is licensed under the CC-NC-ND Creative Commons license.


% start a document with the here given default font size and paper size
\documentclass[10pt,a4paper]{article}

% include the `tex` instructions that takes care of loading packages and defining commands
% Copyright (c) 2012 Cies Breijs
%
% The MIT License
%
% Permission is hereby granted, free of charge, to any person obtaining a copy
% of this software and associated documentation files (the "Software"), to deal
% in the Software without restriction, including without limitation the rights
% to use, copy, modify, merge, publish, distribute, sublicense, and/or sell
% copies of the Software, and to permit persons to whom the Software is
% furnished to do so, subject to the following conditions:
%
% The above copyright notice and this permission notice shall be included in
% all copies or substantial portions of the Software.
%
% THE SOFTWARE IS PROVIDED "AS IS", WITHOUT WARRANTY OF ANY KIND, EXPRESS OR
% IMPLIED, INCLUDING BUT NOT LIMITED TO THE WARRANTIES OF MERCHANTABILITY,
% FITNESS FOR A PARTICULAR PURPOSE AND NONINFRINGEMENT. IN NO EVENT SHALL THE
% AUTHORS OR COPYRIGHT HOLDERS BE LIABLE FOR ANY CLAIM, DAMAGES OR OTHER
% LIABILITY, WHETHER IN AN ACTION OF CONTRACT, TORT OR OTHERWISE, ARISING FROM,
% OUT OF OR IN CONNECTION WITH THE SOFTWARE OR THE USE OR OTHER DEALINGS IN THE
% SOFTWARE.


% Some commands for making a LaTeX resume
% =======================================

% Commented ;)

% See the README.md file for more info



% \documentclass[10pt,a4paper]{article}  % i do this in the document itself


%%% LOAD AND SETUP PACKAGES

\usepackage[a4paper,margin=0.75in]{geometry}
\usepackage{mdwlist}   % to finetue lists with a inline heading and indented content (see Experiences)
\usepackage{multicol}  % for multiple column text
\usepackage{relsize}   % for \textscale, which I prefer over \sc (small caps), see my \acr command
\usepackage[english]{babel}
\hyphenation{Some-long-word}

\usepackage{hyperref}  % yups, URLs everwhere...
\usepackage{xcolor}  % ... and color them links
\definecolor{dark-blue}{rgb}{0.15,0.15,0.4}
\hypersetup{colorlinks,linkcolor={dark-blue},citecolor={dark-blue},urlcolor={dark-blue}}

\usepackage{ifxetex}
\ifxetex
  \usepackage{fontspec}
%  \setmainfont
%    [ ExternalLocation ,
%      Mapping          = tex-text ,
%      Numbers          = OldStyle ,
%      Ligatures        = {Common,Contextual} ,
%      BoldFont         = texgyrepagella-bold.otf ,
%      ItalicFont       = texgyrepagella-italic.otf ,
%      BoldItalicFont   = texgyrepagella-bolditalic.otf ]
%    {texgyrepagella-regular.otf}
  % Comment out the previous statement and uncomment the following line to use the
  % Linux Libertine font (it has nice lignatures).
  % Make sure to have the `ttf-linux-libertine` package installed on Ubuntu.
\setmainfont[Mapping=tex-text, Numbers=OldStyle, Ligatures={Common,Contextual}]{Linux Libertine Display}
  \usepackage[protrusion]{microtype}  % needs an experimental and impposible to find package for xetex
\else
  \usepackage{tgpagella}  % this case we lack lower case numbers, ligatures and some typographic niceties
  \usepackage[expansion,protrusion]{microtype}
\fi



%%% DOCUMENT WIDE STYLING

\pagestyle{empty}
\setlength{\tabcolsep}{0em}
\xspaceskip7pt  % some more spacing between sentences (use "i.e.\ " or "with SQL\@. " in case of errors)


%%% CUSTOM COMMANDS

% main title (name) with subtitle (date)
\newcommand*\maintitle[2]{\noindent{\LARGE \textbf{#1}}\ \ \ \emph{#2}}

% title for the root sections (experience, education, etc) of the resume
\newcommand*\roottitle[1]{\subsection*{#1}\vspace{-0.3em}\nopagebreak[4]}

% acr command, to quickly mark acronyms for special formatting
\newcommand*\acr[1]{\textscale{.85}{#1}}

% pretty bullet (created from a much smaller centerdot), \sbull contains its spacing
\newcommand*\bull{\raisebox{-0.365em}[-1em][-1em]{\textscale{4}{$\cdot$}}}
\newcommand*\sbull{\ \ \bull \ \ }

% it seems not to work when simply using \parindent...
\newlength{\newparindent}
\addtolength{\newparindent}{\parindent}

% a double \parindent...
\newlength{\doubleparindent}
\addtolength{\doubleparindent}{\parindent}
\addtolength{\doubleparindent}{\parindent}

% indentsection style, used for sections that aren't already in lists
% that need indentation to the level of all text in the document
\newenvironment{indentsection}%
{\begin{list}{}%
  {\setlength{\leftmargin}{\newparindent}\setlength{\parsep}{0pt}\setlength{\parskip}{0pt}\setlength{\itemsep}{0pt}\setlength{\topsep}{0pt}}%
}
{\end{list}}

% headerrow command, used for a new employer
\newcommand{\headedsection}[3]{\nopagebreak[4]\begin{indentsection}\item[]\textscale{1.1}{#1}\hfill#2#3\end{indentsection}\nopagebreak[4]}

% subheaderrow command, used for a new position
\newcommand{\headedsubsection}[3]{\nopagebreak[4]\begin{indentsection}\item[]\textbf{#1}\hfill\emph{#2}#3\end{indentsection}\nopagebreak[4]}

% body text (indented)
\newcommand{\bodytext}[1]{\nopagebreak[4]\begin{indentsection}\item[]#1\end{indentsection}\pagebreak[2]}

% \vspace variaties
\newcommand{\breakvspace}[1]{\pagebreak[2]\vspace{#1}\pagebreak[2]}
\newcommand{\nobreakvspace}[1]{\nopagebreak[4]\vspace{#1}\nopagebreak[4]}

% \spacedhrule a horizontal line with some vertical space before and after it
\newcommand{\spacedhrule}[2]{\breakvspace{#1}\hrule\nobreakvspace{#2}}

% \inlineheadsection command, used for a new employer
\newcommand{\inlineheadsection}[2]{\begin{basedescript}{\setlength{\leftmargin}{\doubleparindent}}\item[\hspace{\newparindent}\textbf{#1}]#2\end{basedescript}\vspace{-1.7em}}

% apo command, for an apostrophe that looks good on old style nums
\newcommand{\apo}{\raisebox{-.18ex}{'}{\hspace{-.1em}}}

% non space that allows line breaks
\newcommand*{\nsp}{\hskip0pt}

%%% MORE SPECIFIC COMMANDS

% CPP command (found it in some corner of the internet and decided to use it)
\newcommand{\CPP}{C\nolinebreak[4]\hspace{-.04em}\raisebox{.20ex}{\footnotesize\bf++} }

% KTurtle command :)
\newcommand{\KTurtle}{\acr{KT}urtle }



% % these are in the document itself:
%
% \begin{document}
% ...the document text...
% \end{document}




\begin{document}  % begin the content of the document
\sloppy  % this to relax whitespacing in favour of straight margins

\maintitle{Arjun Naik}{October 29, 1987}  % title on top of the document

\nobreakvspace{0.3em}  % add some page break averse vertical spacing

% \noindent prevents paragraph's first lines from indenting
% \mbox is used to obfuscate the email address
% \sbull is a spaced bullet
% \href well..
% \\ breaks the line into a new paragraph
\noindent\href{mailto:Arjun.RN@gmail.com}{Arjun.RN\mbox{}@\mbox{}gmail.com}\sbull
\textsmaller{+}49.17657886406
\sbull arjun-naik \emph{(Skype)}
\\
\sbull
\href{https://de.linkedin.com/in/arnaik}{https://de.linkedin.com/in/arnaik}
\sbull
\href{https://github.com/arjunrn}{https://github.com/arjunrn}
\\
Berliner Allee 148\sbull
13088\sbull
Berlin\sbull
Germany

\spacedhrule{0.9em}{-0.4em}  % a horizontal line with some vertical spacing before and after

\roottitle{Summary}  % a root section title

\vspace{-1.3em}  % some vertical spacing
\begin{multicols}{2}  % open a multicolumn environment
\noindent \emph{A software/devops engineer with a focus on reliability and performance of distributed systems.}
\\
In 2010 after completing a Bachelor's degree in Computer Science I started working as a software Developer in a startup. I worked with diverse technologies and built applications for several platforms. But I always enjoyed working on the automation and infrastructure for those application. In 2012 I decided to continue my education in Germany. Studying at TU Dresden in the Systems Engineering chair gave me a unique opportunity to participate is several research projects and revisit the fundamentals of Distributed Sytems. After graduating in 2015 I decided to build my career as a Site Reliabiltiy Engineer. My goal is to work in an environment which fosters creativity and offers interesting challenges. Since I am a big proponent of open-source, I also wish that my work both utilises and contributes back to the opensource community.

\end{multicols}

\spacedhrule{0em}{-0.4em}

\roottitle{Experience}

\headedsection
{\href{https://zalando.de/}{Zalando SE}}
{\textsc{Berlin, Germany}}
{
 \headedsubsection
 {\acr{Software Engineer (AWS and Kubernetes) }}
 {Aug 2018 - Current}
 {\bodytext{
    Provide Kubernetes clusters and CI/CD integration to feature developments team in Zalando to deploy their applications. Monitoring and upgrading the clusters
    so that teams can benefit from improved security and reliability. Also develop internal and opensource tooling for cluster users to deploy their applications
    seamlessly and perform application deployments with advanced features like pre-scaling, blue-green deployments and easy rollbacks. Improving the pod autoscaling
    controllers in the clusters so that user workloads can scale both vertically and horizontally. Occasional contributions to upstream Kubernetes open source projects.
    Constantly improving our cluster management tools so that we can roll out upgrades to clusters reliably and with no impact to users.
	}
	}
}

\headedsection
{\href{https://cubits.com/}{Cubits}}
{\textsc{Berlin, Germany}}
{
 \headedsubsection
 {\acr{Site Reliability Engineer}}
 {Jan 2018 - May 2018}
 {\bodytext{
    Cubits is a cryptocurrency payment gateway. I am part of a team which is migrating the existing stack to the cloud in GCP. Existing services are 
    being modified to run in Kubernetes. I am also assiting the feature team in infrastructure for new cryptocurrencies.
	}
	\textbf{Technologies Used:} Google Compute Platform, Terraform, Helm, Prometheus.
	}
}

\headedsection
{\href{http://tech.zalando.com/}{Zalando SE}}
{\textsc{Berlin, Germany}}
{
 \headedsubsection
 {\acr{Site Reliability Engineer}}
 {August 2015 - December 2017}
 {\bodytext{
      Zalando SE is the largest e-commerce retailer in Europe. I was hired as part of the SRE team which would assist the feature teams in the migration of a monolithic codebase to microservices in the cloud. During this process of migration we assisted teams in automating their deployments, instrumenting their applications for monitoring and performed load test to verify that the newly built services could satisy the performance and scaling requirements. 
      
      I was later moved to the Personalization department where I worked with the Recommendations team to deliver personalized item recommendations with strong Service Level Objectives. I continued automating the deployments of the their infrastructure and wrote other automation to manage their datastores like Elasticsearch and Cassandra. I also introduced monitoring for visualizing and observing the performance of various stages of the recommendations pipeline.
	}
	\textbf{Technologies Used:} AWS, locust, Spring Boot, Kubernetes, Elasticsearch, Cassandra.
	}
}

\headedsection
{\href{http://tu-dresden.de/die_tu_dresden/fakultaeten/fakultaet_informatik/sysa/se?set_language=en&cl=en}{TU Dresden}}
{\textsc{Dresden, Germany}}
{
 \headedsubsection
 {\acr{Wissenschaftliche Hilfskraft}}
 {Mar 2014 - Dec 2014}
 {\bodytext{
As a part of the {\href{http://tu-dresden.de/die_tu_dresden/fakultaeten/fakultaet_informatik/sysa/se/srex}{SREX(Secure Remote Execution)}} project I developed a translator for C programs which added redundancy to regular x86 instructions to detect and correct transient errors in memory and computation.

I also worked with consensus services like ZooKeeper and Consul to evalute their runtime characteristics. I also implemented several new prototype features for a ZooKeeper client library.
	}
	\textbf{Technologies Used:} Python, Fabric, Apache ZooKeeper, C/C++, Scons, Consul, Mercurial
	}
}

\headedsection
{\href{http://dmfs.org/}{DMFS}}
{\textsc{Dresden, Germany}}
{
 \headedsubsection
 {\acr{Application Developer}}
 {Jan 2013 - March 2014}
 {\bodytext{
    My primary role was as a Python backend developer. I developed a web based platform for the bulk retail of mobile apps. The service used OpenSSL to sign packaged apps and verified the source of the app sold through the platform. Also developed the Android SDK which integrates with existing third-party apps to provide this functionality. This also included the task of building and maintaining the infrasturcture for deploying and running the platform. 
	}
	\textbf{Technologies Used:} Python, Android, Django, OpenSSL, PostgreSQL, Git
	}
}

\headedsection
{\href{http://www.notiphi.com}{Locus Labs}}
{\textsc{Bangalore, India}}
{
 \headedsubsection
 {\acr{Web and Mobile Developer}}
 {Jun 2011 - August 2012}
 {\bodytext{
  Locus Labs was an early stage startup for providing location based advertising and services through mobile applications. Being one of the first developers meant I had to manage several resposibilities, from building the backend services, tinkering with the frontend code and also maintaing the deployment pipelines to make releases.
 }
	\textbf{Technologies Used:} Django, Python, Android, Tornado, Fabric, Scrapy, Javascript, JQuery, MySQL, MongoDB, Redis, Git
	}
}

\headedsection  % sets the header for the section and includes any subsections
{\href{http://www.netbramha.com}{Netbramha Studios}}
{\textsc{Bangalore, India}}
{
  \headedsubsection  % sets the header for a subsection and contains usually body text
    {\acr{Web Developer}}
    {Oct 2010 - May 2011}
    {\bodytext{Netbramha Studios is a design consultancy which develops aesthetic and content intensive websites. I was hired to handle the technical aspects: Maintenance of websites and minor front-end development. I also built websites using both third-party and in-house server frameworks. Used JQuery for front-end development and also did some basic layout and styling with \acr{HTML+CSS}.

    }
    \textbf{Technologies Used:} PHP, CodeIgniter, CakePHP, JQuery, Javascript, Python, Django, SVN, Git}
}

\begin{center}
  \emph{Please consult my \href{https://de.linkedin.com/in/arnaik}{LinkedIn Profile} for references.}
\end{center}

\spacedhrule{-0.2em}{-0.4em}

\roottitle{Education}

\headedsection
  {Techniche Universit{\"a}t Dresden}
  {\textsc{Dresden, Germany}} {%
  \headedsubsection
    {Master of Science, Distributed Systems Engineering}
    {2012 -- 2014}
    {\bodytext{
        Completed courses in Security and Cryptography, Software Fault Tolerance and Dependable Systems. Also worked in the Systems Engineering Lab and Software Fault Tolerance Lab. Under the guidance of Prof. Dr. Christof Fetzer completed my Master's Thesis in scaling Zookeeper. The thesis proposed a solution which partitioned the Zookeeper FS namespace to shard the data and increase the throughput by reducing the overhead of the consensus algorithm.}
    }
}

\headedsection
  {BM Sreenivasaiah College of Engineering, Vivesvararya Technological University}
  {\textsc{Bangalore, India}} {%
  \headedsubsection
    {Bachelor of Computer Science and Engineering}
    {2006 -- 2010}
    {\bodytext{
        Major in Computer Science with a focus on Computer Architecture. Was also an editor of the departmental newsletter and a regular contributor. For my final year project developed a Wi-Fi enabled robot which could be controlled over the Internet.}
    }
}

\spacedhrule{1.6em}{-0.4em}

\roottitle{Skills}

\inlineheadsection  % special section that has an inline header with a 'hanging' paragraph
  {Technical Skills:}
    {Since I come from a Software Development background I have extensive experience with programming in Python. I have also developed and maintained large Java backend services. I have been using Go for some time but consider myself a beginner. For the past of couple of years I have been building infrasture in AWS. More recently I have started working with Kubernetes and Terraform for deployment of 12-factor apps. I have used several continuous delivery tools like Jenkins, Gitlab-CI and GoCD for building deployment pipelines and other automation tasks.}

\end{document}
